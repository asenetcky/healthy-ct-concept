% Options for packages loaded elsewhere
\PassOptionsToPackage{unicode}{hyperref}
\PassOptionsToPackage{hyphens}{url}
\PassOptionsToPackage{dvipsnames,svgnames,x11names}{xcolor}
%
\documentclass[
  letterpaper,
  DIV=11,
  numbers=noendperiod]{scrreprt}

\usepackage{amsmath,amssymb}
\usepackage{iftex}
\ifPDFTeX
  \usepackage[T1]{fontenc}
  \usepackage[utf8]{inputenc}
  \usepackage{textcomp} % provide euro and other symbols
\else % if luatex or xetex
  \usepackage{unicode-math}
  \defaultfontfeatures{Scale=MatchLowercase}
  \defaultfontfeatures[\rmfamily]{Ligatures=TeX,Scale=1}
\fi
\usepackage{lmodern}
\ifPDFTeX\else  
    % xetex/luatex font selection
\fi
% Use upquote if available, for straight quotes in verbatim environments
\IfFileExists{upquote.sty}{\usepackage{upquote}}{}
\IfFileExists{microtype.sty}{% use microtype if available
  \usepackage[]{microtype}
  \UseMicrotypeSet[protrusion]{basicmath} % disable protrusion for tt fonts
}{}
\makeatletter
\@ifundefined{KOMAClassName}{% if non-KOMA class
  \IfFileExists{parskip.sty}{%
    \usepackage{parskip}
  }{% else
    \setlength{\parindent}{0pt}
    \setlength{\parskip}{6pt plus 2pt minus 1pt}}
}{% if KOMA class
  \KOMAoptions{parskip=half}}
\makeatother
\usepackage{xcolor}
\setlength{\emergencystretch}{3em} % prevent overfull lines
\setcounter{secnumdepth}{5}
% Make \paragraph and \subparagraph free-standing
\makeatletter
\ifx\paragraph\undefined\else
  \let\oldparagraph\paragraph
  \renewcommand{\paragraph}{
    \@ifstar
      \xxxParagraphStar
      \xxxParagraphNoStar
  }
  \newcommand{\xxxParagraphStar}[1]{\oldparagraph*{#1}\mbox{}}
  \newcommand{\xxxParagraphNoStar}[1]{\oldparagraph{#1}\mbox{}}
\fi
\ifx\subparagraph\undefined\else
  \let\oldsubparagraph\subparagraph
  \renewcommand{\subparagraph}{
    \@ifstar
      \xxxSubParagraphStar
      \xxxSubParagraphNoStar
  }
  \newcommand{\xxxSubParagraphStar}[1]{\oldsubparagraph*{#1}\mbox{}}
  \newcommand{\xxxSubParagraphNoStar}[1]{\oldsubparagraph{#1}\mbox{}}
\fi
\makeatother


\providecommand{\tightlist}{%
  \setlength{\itemsep}{0pt}\setlength{\parskip}{0pt}}\usepackage{longtable,booktabs,array}
\usepackage{calc} % for calculating minipage widths
% Correct order of tables after \paragraph or \subparagraph
\usepackage{etoolbox}
\makeatletter
\patchcmd\longtable{\par}{\if@noskipsec\mbox{}\fi\par}{}{}
\makeatother
% Allow footnotes in longtable head/foot
\IfFileExists{footnotehyper.sty}{\usepackage{footnotehyper}}{\usepackage{footnote}}
\makesavenoteenv{longtable}
\usepackage{graphicx}
\makeatletter
\newsavebox\pandoc@box
\newcommand*\pandocbounded[1]{% scales image to fit in text height/width
  \sbox\pandoc@box{#1}%
  \Gscale@div\@tempa{\textheight}{\dimexpr\ht\pandoc@box+\dp\pandoc@box\relax}%
  \Gscale@div\@tempb{\linewidth}{\wd\pandoc@box}%
  \ifdim\@tempb\p@<\@tempa\p@\let\@tempa\@tempb\fi% select the smaller of both
  \ifdim\@tempa\p@<\p@\scalebox{\@tempa}{\usebox\pandoc@box}%
  \else\usebox{\pandoc@box}%
  \fi%
}
% Set default figure placement to htbp
\def\fps@figure{htbp}
\makeatother
% definitions for citeproc citations
\NewDocumentCommand\citeproctext{}{}
\NewDocumentCommand\citeproc{mm}{%
  \begingroup\def\citeproctext{#2}\cite{#1}\endgroup}
\makeatletter
 % allow citations to break across lines
 \let\@cite@ofmt\@firstofone
 % avoid brackets around text for \cite:
 \def\@biblabel#1{}
 \def\@cite#1#2{{#1\if@tempswa , #2\fi}}
\makeatother
\newlength{\cslhangindent}
\setlength{\cslhangindent}{1.5em}
\newlength{\csllabelwidth}
\setlength{\csllabelwidth}{3em}
\newenvironment{CSLReferences}[2] % #1 hanging-indent, #2 entry-spacing
 {\begin{list}{}{%
  \setlength{\itemindent}{0pt}
  \setlength{\leftmargin}{0pt}
  \setlength{\parsep}{0pt}
  % turn on hanging indent if param 1 is 1
  \ifodd #1
   \setlength{\leftmargin}{\cslhangindent}
   \setlength{\itemindent}{-1\cslhangindent}
  \fi
  % set entry spacing
  \setlength{\itemsep}{#2\baselineskip}}}
 {\end{list}}
\usepackage{calc}
\newcommand{\CSLBlock}[1]{\hfill\break\parbox[t]{\linewidth}{\strut\ignorespaces#1\strut}}
\newcommand{\CSLLeftMargin}[1]{\parbox[t]{\csllabelwidth}{\strut#1\strut}}
\newcommand{\CSLRightInline}[1]{\parbox[t]{\linewidth - \csllabelwidth}{\strut#1\strut}}
\newcommand{\CSLIndent}[1]{\hspace{\cslhangindent}#1}

\KOMAoption{captions}{tableheading}
\makeatletter
\@ifpackageloaded{tcolorbox}{}{\usepackage[skins,breakable]{tcolorbox}}
\@ifpackageloaded{fontawesome5}{}{\usepackage{fontawesome5}}
\definecolor{quarto-callout-color}{HTML}{909090}
\definecolor{quarto-callout-note-color}{HTML}{0758E5}
\definecolor{quarto-callout-important-color}{HTML}{CC1914}
\definecolor{quarto-callout-warning-color}{HTML}{EB9113}
\definecolor{quarto-callout-tip-color}{HTML}{00A047}
\definecolor{quarto-callout-caution-color}{HTML}{FC5300}
\definecolor{quarto-callout-color-frame}{HTML}{acacac}
\definecolor{quarto-callout-note-color-frame}{HTML}{4582ec}
\definecolor{quarto-callout-important-color-frame}{HTML}{d9534f}
\definecolor{quarto-callout-warning-color-frame}{HTML}{f0ad4e}
\definecolor{quarto-callout-tip-color-frame}{HTML}{02b875}
\definecolor{quarto-callout-caution-color-frame}{HTML}{fd7e14}
\makeatother
\makeatletter
\@ifpackageloaded{bookmark}{}{\usepackage{bookmark}}
\makeatother
\makeatletter
\@ifpackageloaded{caption}{}{\usepackage{caption}}
\AtBeginDocument{%
\ifdefined\contentsname
  \renewcommand*\contentsname{Table of contents}
\else
  \newcommand\contentsname{Table of contents}
\fi
\ifdefined\listfigurename
  \renewcommand*\listfigurename{List of Figures}
\else
  \newcommand\listfigurename{List of Figures}
\fi
\ifdefined\listtablename
  \renewcommand*\listtablename{List of Tables}
\else
  \newcommand\listtablename{List of Tables}
\fi
\ifdefined\figurename
  \renewcommand*\figurename{Figure}
\else
  \newcommand\figurename{Figure}
\fi
\ifdefined\tablename
  \renewcommand*\tablename{Table}
\else
  \newcommand\tablename{Table}
\fi
}
\@ifpackageloaded{float}{}{\usepackage{float}}
\floatstyle{ruled}
\@ifundefined{c@chapter}{\newfloat{codelisting}{h}{lop}}{\newfloat{codelisting}{h}{lop}[chapter]}
\floatname{codelisting}{Listing}
\newcommand*\listoflistings{\listof{codelisting}{List of Listings}}
\makeatother
\makeatletter
\makeatother
\makeatletter
\@ifpackageloaded{caption}{}{\usepackage{caption}}
\@ifpackageloaded{subcaption}{}{\usepackage{subcaption}}
\makeatother

\usepackage{bookmark}

\IfFileExists{xurl.sty}{\usepackage{xurl}}{} % add URL line breaks if available
\urlstyle{same} % disable monospaced font for URLs
\hypersetup{
  pdftitle={Healthy Connecticut 20XX},
  pdfauthor={Alexander Senetcky},
  colorlinks=true,
  linkcolor={blue},
  filecolor={Maroon},
  citecolor={Blue},
  urlcolor={Blue},
  pdfcreator={LaTeX via pandoc}}


\title{Healthy Connecticut 20XX}
\usepackage{etoolbox}
\makeatletter
\providecommand{\subtitle}[1]{% add subtitle to \maketitle
  \apptocmd{\@title}{\par {\large #1 \par}}{}{}
}
\makeatother
\subtitle{State Health Assessment Concept}
\author{Alexander Senetcky}
\date{2024-12-18}

\begin{document}
\maketitle

\renewcommand*\contentsname{Table of contents}
{
\hypersetup{linkcolor=}
\setcounter{tocdepth}{2}
\tableofcontents
}

\bookmarksetup{startatroot}

\chapter*{Welcome}\label{welcome}
\addcontentsline{toc}{chapter}{Welcome}

\markboth{Welcome}{Welcome}

This is a website for showcasing how the state can adopt to using
markdown, quarto and the Open Data Portal (ODP) to streamline the State
Health Assessment creation process allowing for more timely releases and
in turn providing stakeholders with greater agency and the ability to
act on the information housed therein.

This concept will be \emph{deliberately} using the same or similiar
language from the 2025 Health Assessment for ease of comparison.

To learn more about Quarto books visit
\url{https://quarto.org/docs/books}.

To learn more about the latest CT State Health Assessment visit
\url{https://portal.ct.gov/dph/state-health-planning/healthy-connecticut/healthy-connecticut-2025}

\bookmarksetup{startatroot}

\chapter*{Acknowledgements}\label{acknowledgements}
\addcontentsline{toc}{chapter}{Acknowledgements}

\markboth{Acknowledgements}{Acknowledgements}

\section*{Connecticut Department of Public
Health}\label{connecticut-department-of-public-health}
\addcontentsline{toc}{section}{Connecticut Department of Public Health}

\markright{Connecticut Department of Public Health}

\textbf{J. Smith Commish, MD, MPH } \emph{Acting Commisioner}

\textbf{J. Smith Dep, MPH, LNHA} \emph{Deputy Commisioner}

\begin{center}\rule{0.5\linewidth}{0.5pt}\end{center}

This Connecticut State Health Assessment was developed by the
Connecticut Department of Public Health with the assistance of the
Connecticut Health Improvement Coalition, and its Action Teams and
Advisory Council.

This Assessment is the result of more than a year of dedicated and
collaborative effort of DPH staff, staff from several other State
agencies, and subject matter experts throughout the state who analyzed
and contributed data and reviewed multiple iterations of this document
as it evolved. This Assessment would not have been possible without
their expertise and commitment to this project.

We gratefully acknowledge the contributions of our consultant,

\begin{quote}
\textbf{Health Resources in Action}\\
Boston, MA
\end{quote}

for facilitating collaborative activities of the State Health
Improvement Coalition and its Advisory Council, and for assisting with
compiling this Assessment in cooperation with DPH.

\begin{center}\rule{0.5\linewidth}{0.5pt}\end{center}

\emph{This publication was supported by the Preventive Health \& Health
Services Block Grant (PHHSBG), Grant \# 1NB01OT009192-01-00, funded by
the Centers for Disease Control and Prevention. Its contents are solely
the responsibility of the authors and do not necessarily represent the
official views of the Centers for Disease Control and Prevention or the
Department of Health and Human Services.}

\bookmarksetup{startatroot}

\chapter*{Preface}\label{preface}
\addcontentsline{toc}{chapter}{Preface}

\markboth{Preface}{Preface}

\section*{Letter from the
Commissioner}\label{letter-from-the-commissioner}
\addcontentsline{toc}{section}{Letter from the Commissioner}

\markright{Letter from the Commissioner}

\textbf{To Our Residents and Public Health Partners:}

The Connecticut Department of Public Health is pleased to present the
Healthy Connecticut 2025 State Health Assessment. Subject matter experts
from the Connecticut Department of Public Health (CT DPH), in
collaboration with other state agencies, statewide partners and
community organizations, have assembled data reflecting on the health
and safety of Connecticut residents. The last such document was
published in 2014.

The State Health Assessment establishes the health status of the state,
and will inform the prioritization and development of the next Healthy
Connecticut 2025 State Health Improvement Plan (SHIP). This plan will
serve as a 5-year roadmap for promoting and advancing population health
in our State. Statewide partners from the Connecticut Health Improvement
Coalition, along with CT DPH, will begin the collaborative development
of the SHIP in January 2020.

While Connecticut is a healthy state overall, this assessment highlights
the challenges faced around achieving health equity for all our
residents. The Centers for Disease Control and Prevention (CDC) states
that health equity is achieved when every person has the opportunity to
``attain his or her full health potential'' and no one is
``disadvantaged from achieving this potential because of social position
or other socially determined circumstances.'' CT DPH is committed to
enhancing health equity for our state; this document is an affirmation
that equitable access to healthcare and addressing those social
conditions that impact health is a basic human right.

The Healthy Connecticut 2025 initiative will focus on making the
connection between social determinants and health outcomes. To
experience success with these efforts we must prioritize examining the
impact of social, behavioral and environmental factors on health to
better inform policies and promote systemic change, while exploring
collaborative place-based initiatives with our municipal and local
health partners. It is our hope that we continue to work together to
address the needs of Connecticut residents and afford every single
person the opportunity to be as healthy as possible.

We look forward to collaborating with you in the future on this
important work.

Sincerely,

\textbf{J. Smith Commish, MD, MPH}\\
\emph{Acting Commissioner}

\part{Introduction and Process}

\chapter{Introduction \& Process}\label{introduction-process}

\section{What is the State Health
Assessment?}\label{what-is-the-state-health-assessment}

The 2019 Connecticut State Health Assessment is an update on the health
status of Connecticut residents with a focus on the social determinants
of health that are having the greatest impact on health outcomes. The
assessment provides the basis for the Connecticut State Health
Improvement Plan, which together make up the state health planning
framework Healthy Connecticut 2025.

The purpose of the assessment is to provide the public, policy leaders,
partners, and stakeholders with information on the health of the
Connecticut population to develop a shared understanding of health
issues and inform data-driven decision making and program planning. This
state health assessment is an important tool to help identify the
underlying conditions and factors that influence health, reflect on
existing services and policies, and inform future public health planning
for the benefit of all Connecticut communities.

\section{Visions for Health Equity}\label{visions-for-health-equity}

Connecticut has a bold vision for Healthy Connecticut 2025. More
specifically, the Connecticut Department of Public Health (CT DPH) and
partners envision the following:

\begin{center}\rule{0.5\linewidth}{0.5pt}\end{center}

\begin{quote}
Through effective assessment, prevention, and policy development, the
Connecticut Department of Public Health and its stakeholders and
partners provide every Connecticut resident equitable opportunities to
be healthy throughout their lifetimes and are accountable to making
measurable improvements toward health equity.
\end{quote}

\begin{center}\rule{0.5\linewidth}{0.5pt}\end{center}

This vision lifts up a number of guiding principles that we uphold to
center health equity:

\begin{itemize}
\item
  \textbf{A focus on every Connecticut resident:} We strive for all
  Connecticut residents to experience optimal health and wellbeing.*
\item
  Attention to the health needs of residents \textbf{throughout their
  lifetimes.}
\item
  \textbf{A need to collaborate as stakeholders and partners:} No one
  entity can advance health equity in isolation. A multi-sector and
  community-engaged approach is necessary to effectively understand the
  interconnected social determinants that impact health, and effectively
  address the practices, policies, and systems that support them.
\item
  A multi-pronged approach through \textbf{assessment, prevention,
  policy development and accountability to achieve measurable
  improvements in health equity.}
\end{itemize}

More information about health equity, health disparities, and the Social
Determinants of Health can be found in the Describing Connecticut
chapter.

\section{Methodology}\label{methodology}

The Healthy Connecticut 2025: State Health Assessment was ultimately
guided by Connecticut's vision for health equity. The health indicators
selected to be presented in the assessment reflect the social
determinants of health that are impacting residents and highlight where
health is experienced differently based on geographic or demographic
characteristics.

The development of this assessment incorporated the Mobilizing for
Action through Planning and Partnerships (MAPP) framework and Public
Health Accreditation Board (PHAB) standards and measures. A
cross-disciplinary team of internal and external stakeholders was
engaged to develop a vision for Healthy Connecticut 2025 and to
prioritize a list of health indicators for inclusion in the report. In
addition, community members were provided opportunities to contribute to
the development of the assessment through surveys and focus groups, and
finally through a public comment period.

\subsection{Engagement Process}\label{engagement-process}

Duis urna urna, pellentesque eu urna ut, malesuada bibendum dolor.
Suspendisse potenti. Vivamus ornare, arcu quis molestie ultrices, magna
est accumsan augue, auctor vulputate erat quam quis neque. Nullam
scelerisque odio vel ultricies facilisis. Ut porta arcu non magna
sagittis lacinia. Cras ornare vulputate lectus a tristique. Pellentesque
ac arcu congue, rhoncus mi id, dignissim ligula.

\subsubsection{Community Survey}\label{community-survey}

Lorem ipsum dolor sit amet, consectetur adipiscing elit. Duis sagittis
posuere ligula sit amet lacinia. Duis dignissim pellentesque magna,
rhoncus congue sapien finibus mollis. Ut eu sem laoreet, vehicula ipsum
in, convallis erat. Vestibulum magna sem, blandit pulvinar augue sit
amet, auctor malesuada sapien. Nullam faucibus leo eget eros hendrerit,
non laoreet ipsum lacinia. Curabitur cursus diam elit, non tempus ante
volutpat a. Quisque hendrerit blandit purus non fringilla. Integer sit
amet elit viverra ante dapibus semper. Vestibulum viverra rutrum enim,
at luctus enim posuere eu. Orci varius natoque penatibus et magnis dis
parturient montes, nascetur ridiculus mus.

Nunc ac dignissim magna. Vestibulum vitae egestas elit. Proin feugiat
leo quis ante condimentum, eu ornare mauris feugiat. Pellentesque
habitant morbi tristique senectus et netus et malesuada fames ac turpis
egestas. Mauris cursus laoreet ex, dignissim bibendum est posuere
iaculis. Suspendisse et maximus elit. In fringilla gravida ornare.
Aenean id lectus pulvinar, sagittis felis nec, rutrum risus. Nam vel
neque eu arcu blandit fringilla et in quam. Aliquam luctus est sit amet
vestibulum eleifend. Phasellus elementum sagittis molestie. Proin tempor
lorem arcu, at condimentum purus volutpat eu. Fusce et pellentesque
ligula. Pellentesque id tellus at erat luctus fringilla. Suspendisse
potenti.

\subsubsection{Community Focus Groups}\label{community-focus-groups}

Ut ut condimentum augue, nec eleifend nisl. Sed facilisis egestas odio
ac pretium. Pellentesque consequat magna sed venenatis sagittis. Vivamus
feugiat lobortis magna vitae accumsan. Pellentesque euismod malesuada
hendrerit. Ut non mauris non arcu condimentum sodales vitae vitae dolor.
Nullam dapibus, velit eget lacinia rutrum, ipsum justo malesuada odio,
et lobortis sapien magna vel lacus. Nulla purus neque, hendrerit non
malesuada eget, mattis vel erat. Suspendisse potenti.

Nullam dapibus cursus dolor sit amet consequat. Nulla facilisi.
Curabitur vel nulla non magna lacinia tincidunt. Duis porttitor quam
leo, et blandit velit efficitur ut. Etiam auctor tincidunt porttitor.
Phasellus sed accumsan mi. Fusce ut erat dui. Suspendisse eu augue eget
turpis condimentum finibus eu non lorem. Donec finibus eros eu ante
condimentum, sed pharetra sapien sagittis. Phasellus non dolor ac ante
mollis auctor nec et sapien. Pellentesque vulputate at nisi eu
tincidunt. Vestibulum at dolor aliquam, hendrerit purus eu, eleifend
massa. Morbi consectetur eros id tincidunt gravida. Fusce ut enim quis
orci hendrerit lacinia sed vitae enim.

\subsubsection{Public and Partner Input}\label{public-and-partner-input}

Aenean placerat luctus tortor vitae molestie. Nulla at aliquet nulla.
Sed efficitur tellus orci, sed fringilla lectus laoreet eget. Vivamus
maximus quam sit amet arcu dignissim, sed accumsan massa ullamcorper.
Sed iaculis tincidunt feugiat. Nulla in est at nunc ultricies dictum ut
vitae nunc. Aenean convallis vel diam at malesuada. Suspendisse arcu
libero, vehicula tempus ultrices a, placerat sit amet tortor. Sed dictum
id nulla commodo mattis. Aliquam mollis, nunc eu tristique faucibus,
purus lacus tincidunt nulla, ac pretium lorem nunc ut enim. Curabitur
eget mattis nisl, vitae sodales augue. Nam felis massa, bibendum sit
amet nulla vel, vulputate rutrum lacus. Aenean convallis odio pharetra
nulla mattis consequat.

Ut ut condimentum augue, nec eleifend nisl. Sed facilisis egestas odio
ac pretium. Pellentesque consequat magna sed venenatis sagittis. Vivamus
feugiat lobortis magna vitae accumsan. Pellentesque euismod malesuada
hendrerit. Ut non mauris non arcu condimentum sodales vitae vitae dolor.
Nullam dapibus, velit eget lacinia rutrum, ipsum justo malesuada odio,
et lobortis sapien magna vel lacus. Nulla purus neque, hendrerit non
malesuada eget, mattis vel erat. Suspendisse potenti.

\subsection{Assets and Resources}\label{assets-and-resources}

As CT DPH teams gathered and analyzed data for the assessment, they also
compiled a list of programmatic and state-wide assets. Additionally,
through an analysis of local community health assessments and hospital
health needs assessments, and partner input, community assets were added
to develop a comprehensive list. A high level description of community
assets and resources is available in Appendix B. A more detailed listing
of identified assets, including the analysis of the local community
health and hospital health assessments is contained in a companion
document ``Assets and Resources'' available on the Coalition website.
This document will serve as a dynamic and continuously updated resource
for mapping assets to intentionally developed collaborative strategies.

\section{Kinds of Data Presented in the
Assessment}\label{kinds-of-data-presented-in-the-assessment}

The State Health Assessment presents many kinds of data visualized in
graphs, tables, and maps. Here are some examples of data types included
in this report and what they mean.

\begin{tcolorbox}[enhanced jigsaw, toptitle=1mm, opacityback=0, toprule=.15mm, title=\textcolor{quarto-callout-important-color}{\faExclamation}\hspace{0.5em}{Important}, rightrule=.15mm, bottomtitle=1mm, leftrule=.75mm, left=2mm, colback=white, breakable, opacitybacktitle=0.6, colbacktitle=quarto-callout-important-color!10!white, titlerule=0mm, arc=.35mm, colframe=quarto-callout-important-color-frame, bottomrule=.15mm, coltitle=black]

Please note we can do much prettier tables in code, and in fact I'll
insist on it, but for demonstration purposes, we'll use straight
markdown.

\end{tcolorbox}

\begin{longtable}[]{@{}
  >{\raggedright\arraybackslash}p{(\linewidth - 6\tabcolsep) * \real{0.3158}}
  >{\raggedright\arraybackslash}p{(\linewidth - 6\tabcolsep) * \real{0.2895}}
  >{\raggedright\arraybackslash}p{(\linewidth - 6\tabcolsep) * \real{0.2368}}
  >{\raggedright\arraybackslash}p{(\linewidth - 6\tabcolsep) * \real{0.1579}}@{}}
\toprule\noalign{}
\begin{minipage}[b]{\linewidth}\raggedright
Data Type
\end{minipage} & \begin{minipage}[b]{\linewidth}\raggedright
Answers the Question
\end{minipage} & \begin{minipage}[b]{\linewidth}\raggedright
Definition
\end{minipage} & \begin{minipage}[b]{\linewidth}\raggedright
\end{minipage} \\
\midrule\noalign{}
\endhead
\bottomrule\noalign{}
\endlastfoot
Census & How many people are/ have \_\_\_\_? & An official count & \\
Prevalence & What percentage of people have \_\_\_\_? & describes how
many & \\
Incidence & How many new cases of \_\_\_\_ happened in a period of time?
& this refers to the number & \\
\end{longtable}

Ut ut condimentum augue, nec eleifend nisl. Sed facilisis egestas odio
ac pretium. Pellentesque consequat magna sed venenatis sagittis. Vivamus
feugiat lobortis magna vitae accumsan. Pellentesque euismod malesuada
hendrerit. Ut non mauris non arcu condimentum sodales vitae vitae dolor.
Nullam dapibus, velit eget lacinia rutrum, ipsum justo malesuada odio,
et lobortis sapien magna vel lacus. Nulla purus neque, hendrerit non
malesuada eget, mattis vel erat. Suspendisse potenti.

\section{Limitations of assessment}\label{limitations-of-assessment}

Nunc ac dignissim magna. Vestibulum vitae egestas elit. Proin feugiat
leo quis ante condimentum, eu ornare mauris feugiat. Pellentesque
habitant morbi tristique senectus et netus et malesuada fames ac turpis
egestas. Mauris cursus laoreet ex, dignissim bibendum est posuere
iaculis. Suspendisse et maximus elit. In fringilla gravida ornare.
Aenean id lectus pulvinar, sagittis felis nec, rutrum risus. Nam vel
neque eu arcu blandit fringilla et in quam. Aliquam luctus est sit amet
vestibulum eleifend. Phasellus elementum sagittis molestie. Proin tempor
lorem arcu, at condimentum purus volutpat eu. Fusce et pellentesque
ligula. Pellentesque id tellus at erat luctus fringilla. Suspendisse
potenti.

\section{MAPS!}\label{maps}

Maps are presented in infographic style throughout the report. Below are
three reference maps of Connecticut's counties and towns (1); its major
highways and cities (2); and major waterways (3).

TODO: throw some ggplot or leaflets in here

\part{Describing Connecticut}

\chapter{Introduction}\label{introduction}

\begin{quote}
In order to fully understand the state of Connecticut's health and
health outcomes, it is imperative for this State Health Assessment to
begin by describing our residents by those fundamental sociodemographics
that contribute to certain populations experiencing a greater burden of
ill health; the difference in these health outcomes on a population
level are health disparities. The World Health Organization states that
``what makes societies prosper and flourish can also make people
healthy.'' At a glance it would appear that Connecticut is doing well
from a national perspective; America's Health Rankings 2018 Annual
Report reported that Connecticut is the third healthiest state in the
country. But even when our society thrives there continue to be pockets
of our people who experience worse health outcomes solely because they
identify or pertain to historically underrepresented groups based on but
not exclusive to sex and sexual orientation, gender identity, race,
ethnicity, or age.
\end{quote}

Identifying who is at greatest risk for preventable health conditions is
an important initial step toward identifying relevant health inequities
and supporting health equity. And while these populations are defined by
elements that are immutable, there are other populations of interest ---
immigrants and refugees, veterans, the formerly incarcerated, and people
with mental health disorders --- who also experience poor health
outcomes disparately. Although the 2018 America's Health Rankings Annual
Report found that Connecticut is the third healthiest state in the
nation, we must consider that it is also the most diverse state in New
England; this greater diversity indicates a need for greater resources
in order to respond more equitably. Each of these presents a different
aspect of meeting the health needs of our communities, from having a
competent and diverse workforce to removing language barriers.

Proin sodales neque erat, varius cursus diam tincidunt sit amet. Etiam
scelerisque fringilla nisl eu venenatis. Donec sem ipsum, scelerisque ac
venenatis quis, hendrerit vel mauris. Praesent semper erat sit amet
purus condimentum, sit amet auctor mi feugiat. In hac habitasse platea
dictumst. Nunc ac mauris in massa feugiat bibendum id in dui. Praesent
accumsan urna at lacinia aliquet. Proin ultricies eu est quis
pellentesque. In vel lorem at nisl rhoncus cursus eu quis mi. In eu
rutrum ante, quis placerat justo. Etiam euismod nibh nibh, sed elementum
nunc imperdiet in. Praesent gravida nunc vel odio lacinia, at tempus
nisl placerat. Aenean id ipsum sed est sagittis hendrerit non in tortor.

Lorem ipsum dolor sit amet, consectetur adipiscing elit. Duis sagittis
posuere ligula sit amet lacinia. Duis dignissim pellentesque magna,
rhoncus congue sapien finibus mollis. Ut eu sem laoreet, vehicula ipsum
in, convallis erat. Vestibulum magna sem, blandit pulvinar augue sit
amet, auctor malesuada sapien. Nullam faucibus leo eget eros hendrerit,
non laoreet ipsum lacinia. Curabitur cursus diam elit, non tempus ante
volutpat a. Quisque hendrerit blandit purus non fringilla. Integer sit
amet elit viverra ante dapibus semper. Vestibulum viverra rutrum enim,
at luctus enim posuere eu. Orci varius natoque penatibus et magnis dis
parturient montes, nascetur ridiculus mus.

Where appropriate, this chapter provides comparisons between
Connecticut, the New England region, and the United States.

\chapter{Social Factors}\label{social-factors}

\begin{quote}
The Social Determinants of Health (SDOH) are the upstream non-health
factors that ``impact a wide range of health, functioning and quality of
life outcomes.''1 For public health, this is as preventive as it gets.
When considering these upstream factors in the work of a public health
entity such as the Connecticut Department of Public Health (CT DPH), we
can more effectively inform the public and policymakers so we can all
live better lives. As an agency whose mission declares that the equal
enjoyment of a person's highest attainable standard of health is a human
right, we must also examine the conditions that contribute to
``avoidable differences in health among specific population groups that
result from cumulative social disadvantages.''\footnote{Vavrus, S. J.,
  Obscherning, E., \& Patz, A. (2015). Understanding key concepts of
  climate science and their application. In B. S. Levy \& J. A. Patz
  (Eds.), Climate Change and Public Health. (pp.~29--49). New York:
  Oxford University Press.} More specifically, we apply an equity lens
to ascertain which populations are being most negatively impacted.
\end{quote}

Look references on highlight and at the bottom of the page - go find
that same reference easily in the original document. It's not clickable,
and the pdf is 400 pages long so it's had to find the reference page,
and it takes a while to load. So much friction, now references are
simultanesouly front and center and out of the way.

\section{Education}\label{education}

Economic factors such as poverty and unemployment can lead to unhealthy
living conditions. Yet education can provide individuals with
foundational knowledge, life skills, and social and psychological
supports to make healthier choices. Therefore, quality education and
higher educational attainment can be a protective factor that can
advance more equitable outcomes.3 It has been demonstrated that
individuals without a high school diploma have higher incidences of risk
behaviors and other adverse health outcomes; and earn less money, which
can limit access to resources and healthy environments.4

\subsection{Early Education}\label{early-education}

Experiences and education within the first five years of life can shape
one's health trajectory across the lifespan. Early education and care
programs can be protective against social and economic challenges and
narrow inequitable gaps in health outcomes.5 Participating in these
programs are also associated with higher educational attainment, better
eating habits, increased use of preventive healthcare services, and
lower rates of child injuries, child abuse/maltreatment, teen pregnancy,
depression, use of tobacco or other drugs, and arrests and
incarceration.5 As noted in Figure 1, the rate of Pre-K enrollment for
4-year old children in state-funded preschool programs in Connecticut
has made sizable gains over the past 5 years; however, enrollment among
3-year old children has remained fairly stable over the past decade and
thus far peaked at 10\% in 2016.

\bookmarksetup{startatroot}

\chapter*{References}\label{references}
\addcontentsline{toc}{chapter}{References}

\markboth{References}{References}

\phantomsection\label{refs}
\begin{CSLReferences}{0}{1}
\end{CSLReferences}




\end{document}
